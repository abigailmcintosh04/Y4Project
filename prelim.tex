\documentclass[12pt]{article}
\usepackage{graphicx}
\usepackage[a4paper, margin=2cm]{geometry}
\usepackage{hyperref}

\begin{document}

\begin{titlepage}
    \begin{center}
        \vspace*{1cm}
        \Huge
        \textbf{Exploring Heavy Quark Fragmentation with Machine Learning at ATLAS}
        
        \vspace{0.5cm}
        \LARGE
        Preliminary Report
            
        \vspace{1.5cm}
        \textbf{Abigail McIntosh}
        
        \vfill
        \vspace{0.8cm}
            
            
        \Large
        School of Physics and Astronomy\\
        University of Birmingham\\
        December 2025\\
    \end{center}
\end{titlepage}

\pagenumbering{roman}
\tableofcontents

\pagebreak
\pagenumbering{arabic}


\section{Introduction}

\subsection{Perturbative and Non-Perturbative QCD}
Quantum Chromodynamics (QCD) is the quantum theory that describes the strong interaction between quarks, that is mediated by gluons. A key difference between QCD and Quantum Electrodynamics (QED) is that QCD is non-Abelian gauge theory, which leads to gluons carrying the colour charge. This also means that gluons are self-coupling. This self-coupling leads to the two ranges of QCD: asymptotic freedom and confinement. Under asymptotic freedom at high energies, the strong coupling constant $\alpha_s \ll 1$, meaning perturbative calculations (pQCD) can be used. Under confinement at low energies, $\alpha_s \sim 1$, meaning a non-perturbative approach is necessary. 

For pp collisions at the LHC, the hard scattering process occurs at high energies, meaning pQCD can be used to describe it. However, the fragmentation occurs at low energies, meaning pQCD cannot be used to describe it, and are instead described by phenomenological models. The two approaches required to describe the collisions are connected by the factorisation theorem, which states that the cross section can be split into three parts: the parton distribution functions (PDFs), the hard scatter, and the fragmentation functions.

\subsection{Fragmentation and Universality}
Fragmentation is the non-perturbative process by which coloured partons, i.e. quarks and gluons, are confined into colourless hadrons. The fragmentation function $D^h_i(x, \mu^2) (i=q,\bar{q},g)$ describes the probability density that a certain outgoing parton $i$ produces a hadron $h$, where $x$ is the fraction of the parton's momentum transferred to the hadron and $\mu$ is the factorisation scale \cite{pdg2018}.

Historically, it has been assumed fragmentation functions are universal, meaning they are independent of the hard scattering process. This means fragmentation functions should be the same for pp collisions as for e$^+$e$^-$ collisions. However, recent measurements of charm quark hadronisation in the ALICE collaboration found a larger fraction hadronising into baryons in pp collisions compared to e$^+$e$^-$ collisions \cite{alice2023charm}, suggesting fragmentation may not be universal and instead dependent on the environment.


\section{Event Generation}
To train the machine learning model, Monte Carlo (MC) simulated events are used. The events are generated using PYTHIA 8 \cite{pythia8}, generating pp collisions at $\sqrt{s} = 13$ TeV. To ensure only charm events are generated, the enabled hard QCD processes are qq $\rightarrow$ c\={c} and gg $\rightarrow$ c\={c}. A minimum transverse momentum of 20 GeV is set for the hard scatter to ensure the events are jet-like in nature. Initial and final state radiation and hadronisation are enabled, while multiple parton interactions are disabled. 

For each final state particle in a PYTHIA event, a FastJet \cite{fastjet} PseudoJet object is created. The pseudojets are then clustered using the anti-$k_t$ algorithm, with a radius parameter of R = 0.4, to form jets.+ The jets are then matched to the charm quarks via a $\Delta R$ matching, where $\Delta R = \sqrt{(\Delta \eta)^2 + (\Delta \phi)^2}$, with a maximum $\Delta R$ of 0.4. 

\subsection{Lund String Model}
PYTHIA uses the Lund string model to simulate hadronisation. In this model, when a quark-antiquark pair is separated, the gluon field lines collapse into a flux tube, or string. The potential energy stored in the string increases linearly with the separation distance, $V(r) = \kappa r$, where $\kappa$ is the string tension, approximately 1 GeV/fm. When it becomes energetically favourable, the string will break, creating a new quark-antiquark pair. The probability of creating a pair of mass $\mu$ with transverse momentum $p_{\perp}$ is given by:

\begin{equation}
    P(m) \propto \exp\left(-\frac{\pi (\mu^2 + p_{\perp}^2)}{\kappa}\right),
\end{equation}

meaning the production of heavier quarks (charm and bottom) is suppressed \cite{lundmodel}. Therefore, the production of $\Lambda_c$ baryons (quark content udc) is suppressed compared to D mesons (containing one charm quark and another light quark). The mass of a ud-\={u}\={d} diquark is much larger than a u\={u} or d\={d} pair, with the suppression of diquark production to quark production in PYTHIA defined as 0.081 \cite{pythia8}, where universality is assumed.



\section{Future Plans}

\subsection{Detector Simulation and Realism}
To further improve the realism of the event generator, a detector will be simulated, possibly using DELPHES. This will allow detector effects, such as efficiency, resolution (via a Gaussian smear) and rejection of high pseudorapidity events, to be included in the simulation. This will improve the realism of the event generation, leading to more accurate training of the machine learning model.

\bibliographystyle{plain}
\bibliography{references}

\end{document}