% !TeX program = latexmk
% !TeX outputDirectory = build
\documentclass[12pt]{article}
\usepackage{graphicx}
\usepackage[a4paper, margin=2cm]{geometry}
\usepackage{hyperref}
\usepackage{subcaption}

% --- SPACE SAVING PACKAGES & SETTINGS --- %
\usepackage{titlesec} 
% Reduce space around sections {left}{before}{after}
\titlespacing\section{0pt}{12pt plus 4pt minus 2pt}{6pt plus 2pt minus 2pt}
\titlespacing\subsection{0pt}{10pt plus 4pt minus 2pt}{4pt plus 2pt minus 2pt}

% Reduce space around figures and tables
\setlength{\textfloatsep}{10pt plus 1.0pt minus 2.0pt}
\setlength{\floatsep}{10pt plus 1.0pt minus 2.0pt}
\setlength{\intextsep}{10pt plus 1.0pt minus 2.0pt}

% Reduce space around equations
\AtBeginDocument{
  \setlength{\abovedisplayskip}{6pt plus 2pt minus 2pt}
  \setlength{\belowdisplayskip}{6pt plus 2pt minus 2pt}
}
% ---------------------------------------- %

\begin{document}

\begin{titlepage}
    \begin{center}
        \vspace*{1cm}
        \Huge
        \textbf{Exploring Heavy Quark Fragmentation with Machine Learning at ATLAS}
        
        \vspace{0.5cm}
        \LARGE
        Preliminary Report
        
        \vspace{1.5cm}
        \large
        \textbf{Author: } Abigail McIntosh \\
        \textbf{Project Partner:} Sinead Leckey \\
        \textbf{Supervisors:} Dr Andrew Chisholm, Dr Eleni Skorda \\

        \vspace{1.0cm}
        \textbf{Abstract}

        \vspace{0.2cm}
        \begin{minipage}{0.9\textwidth}
            \small
            [Abstract Placeholder] Lorem ipsum dolor sit amet, consectetur adipiscing elit. Sed do eiusmod tempor incididunt ut labore et dolore magna aliqua. Ut enim ad minim veniam, quis nostrud exercitation ullamco laboris nisi ut aliquip ex ea commodo consequat.
        \end{minipage}
        \vfill            

            
        \Large
        School of Physics and Astronomy\\
        University of Birmingham\\
        December 2025\\
    \end{center}
\end{titlepage}

\pagenumbering{roman}

\pagebreak
\pagenumbering{arabic}

% ---------- INTRODUCTION AND MOTIVATIONS ---------- %

\section{Introduction and Motivations}

\subsection{QCD, Fragmentation and Universality}
Quantum Chromodynamics (QCD) is the quantum theory describing the strong interaction between quarks, mediated by gluons. The self-coupling of gluons leads to two regimes: asymptotic freedom at high energies, where perturbative QCD (pQCD) is applicable, and confinement at low energies, requiring non-perturbative approaches. For pp collisions at the LHC, pQCD can describe the hard scattering, whereas fragmentation has to be modelled pheonomenologically due to the low energies involved. 

Fragmentation is the non-perturbative process by which coloured partons are confined into colourless hadrons. The fragmentation function descrbes the probability density that a certain parton produces a hadron based on the fraction of momentum transferred \cite{pdg2018}. Historically, it has been assumed fragmentation functions are universal, meaning they are independent of the hard scattering process and should be identical for pp and e$^+$e$^-$ collisions. However, recent measurements of charm quark hadronisation at the ALICE collaboration found a larger fraction hadronising into baryons in pp collisions compared to e$^+$e$^-$ collisions \cite{alice2023charm}, suggesting fragmentation may not be universal and instead dependent on the environment.

\subsection{Project Aims}
The aim of this project is to explore heavy quark fragmentation using a machine learning (ML) approach, focusing on charm quarks. An ML model will be created to distinguish between jets containing different charmed hadrons, specifically $D^0$, $D^+$, $D_s^+$ mesons, $\Lambda_c^+$ baryons, and their corresponding antiparticles. An ML-based approach is chosen as traditional methods using reconstruction have low efficiency due to the need to identify all the charm hadron's decay products. The model will be trained on Monte Carlo (MC) simulated events, and then be applied to real data from the ATLAS detector at the LHC. By comparing the ratio of charm hadrons identified in the real data to the MC simulated data, the universality of the charm quark fragmentation can be investigated.

\subsection{Motivations}
As mentioned earlier, phenomenological models are needed to describe fragmentation, therefore these models need their parameters tuned to match experimental data. By improving the efficiency of identifying charm hadrons, the accuracy of Monte Carlo generators such as PYTHIA can be improved.

For a Higgs mass of 125 GeV, the predicted branching fraction of $H \rightarrow c \bar{c}$ is 3\% \cite{higgs}. This decay is yet to be detected, due to its rarity, similarity to other events and domination of background process. In order to detect this decay, charm jets need to be tagged with a high accuracy. The jet tagging algorithms are typically trained on MC simulated data, which typically assume universality. Hence, if fragmentation is not universal in the ATLAS data, the jet tagging algorithms are not being trained correctly, and so would perform worse when applied to real data. By developing an ML model to identify charm hadrons, a comparison between MC and ATLAS data can be made, testing the assumption of universality.

% ---------- EVENT GENERATION ---------- %

\section{Event Generation}
To train the ML model, MC simulated events are used. The events are generated using PYTHIA 8 \cite{pythia8}, generating pp collisions at $\sqrt{s} = 13$ TeV. To ensure only charm events are generated, the enabled hard QCD processes are qq $\rightarrow$ c\={c} and gg $\rightarrow$ c\={c}. A minimum transverse momentum of 20 GeV is set for the hard scatter to ensure the events are jet-like in nature. Initial and final state radiation and hadronisation are enabled, while multiple parton interactions are disabled. 

For each final state particle in a PYTHIA event, a FastJet \cite{fastjet} PseudoJet object is created. The pseudojets are then clustered using the anti-$k_t$ algorithm, with a radius parameter of R = 0.4, to form jets. The jets are then matched to the charm quarks by finding the jet with the smallest $\Delta R$ compared to the charm quark axis, where $\Delta R = \sqrt{(\Delta \eta)^2 + (\Delta \phi)^2}$, with a maximum $\Delta R$ of 0.4 set. 

\subsection{Lund String Model}
PYTHIA uses the Lund string model to simulate hadronisation \cite{pythia8}. In this model, when a quark-antiquark pair is separated, the gluon field lines collapse into a string. When it becomes energetically favourable, the string will break, creating a new quark-antiquark pair. The probability of creating a quark pair is related to the quarks' masses, hence the production of charm and bottom quarks is suppressed \cite{lundmodel}. Therefore, the production of $\Lambda_c$ baryons is suppressed compared to D mesons due to the need to create a ud-\={u}\={d} diquark. The suppression of diquark production to quark production in PYTHIA defined as 0.081 \cite{pythia8}, where universality is assumed.

\subsection{Event Structure}
Each generated event contains multiple jets, however only the constituents of the jet matched to the charm quark are used to calculate the summary variables. The structure of a typical event in eta-phi space can be seen in Figure \ref{fig:eta_phi}, where the charm hadrons are outlined and the jet constituents are scaled by their transverse momentum. A minimum $p_T$ of 10 GeV is applied so only high $p_T$ jets are shown. As can be seen, any given event can include jets containing charmed hadrons, charmed hadrons not matched to any jets, and jets not containing any charmed hadrons.

\subsection{ATLAS Detector}
The ATLAS detector is a particle detector at the LHC. It consists of four main layers: the inner detector, the electromagnetic and hadronic calorimeters, and the muon spectrometer \cite{atlas2008}. The inner detector is responsible for tracking charged particles, meaning the primary and secondary vertices can be reconstructed and used for calculating variables. The calorimeters measure the energy of particles, allowing particle four-momenta to be reconstructed. Hence, the distinguishing variables used will be based entirely on these available measurements to ensure the ML model can be applied to the real data from ATLAS. 

% ---------- MACHINE LEARNING ---------- %

\section{Machine Learning}
To explore the fragmentation of charm quarks, an ML model will be developed to classify jets based on the charmed hadron(s) they contain. The current approach of reconstruction requires identifying all the decay products of the charmed hadron, leading to low reconstruction efficiency. To solve this, an ML model will classify jets based on their overall properties, increasing efficiency. This model will be trained with the Monte Carlo simulated events described in Section 2.

\begin{figure}[ht]
    \centering
    \begin{minipage}{0.4\textwidth}
        \centering
        \includegraphics[width=\linewidth]{images/vertices.png}
        \caption{Diagram showing a typical decay of a charmed hadron at ATLAS, showing the primary and secondary vertices, as well as $L_{xy}$ and $d_0$ \cite{atlas2021triggers}.}
        \label{fig:vertices}
    \end{minipage}\hfill
    \begin{minipage}{0.56\textwidth}
        \centering
        \includegraphics[width=\linewidth]{images/eta_phi_new.png}
        \caption{Eta-phi plot showing jet constituents for a single event, with charmed hadrons outlined in black. A minimum $p_T$ of 10GeV is applied.}
        \label{fig:eta_phi}
    \end{minipage}
\end{figure}

\subsection{Input Features}
As the charmed hadrons have different lifetimes, masses, and decay properties (as shown in Table \ref{tab:charm_props}), the jets containing them are expected to have slightly different properties. Distinguishing variables are calculated using the available ATLAS detector data as mentioned in Section 2.3 to ensure the model can be applied to the ATLAS data.

% Reduced table spacing
\begin{table}[ht]
    \centering
    \renewcommand{\arraystretch}{0.9} % Make table slightly more compact
    \begin{tabular}{|c|c|c|c|}
        \hline
        Hadron & Quark Content & Mass (MeV) & Lifetime (fs) \\
        \hline 
        $D^0$ & $c\bar{u}$ & $1864.84 \pm 0.05$ & $410.3 \pm 1.0$ \\
        $D^+$ & $c\bar{d}$ & $1869.5 \pm 0.4$ & $1033 \pm 5$ \\
        $D_s^+$ & $c\bar{s}$ & $1969.0 \pm 1.4$ & $501.2 \pm 2.2$ \\
        $\Lambda_c^+$ & $udc$ & $2286.46 \pm 0.14$ & $202.6 \pm 1.0$ \\
        \hline
    \end{tabular}
    \caption{Properties of charmed hadrons being used \cite{pdg}.}
    \label{tab:charm_props}
\end{table}

Two main distinguishing variables that have been identified so far are $L_{xy}$, which is the transverse decay length of the charmed hadron, and $d_0$, which is the transverse impact parameter of the charmed hadron. They are defined as follows:
\begin{equation}
    L_{xy} = \sqrt{(x_{SV}-x_{PV})^2 + (y_{SV}-y_{PV})^2},
\end{equation}
\begin{equation}
    d_0 = \frac{1}{p_T} (x_{PV} p_y - y_{PV} p_x),
\end{equation}
where ($x_{PV}$, $y_{PV})$ and $(x_{SV}$, $y_{SV})$ are the x and y coordinates of the primary vertex (PV) and secondary vertex (SV) respectively, and $p_T$, $p_x$ and $p_y$ are the transverse momentum, x component and y component of the charmed hadron  respectively.

It is important to note that $L_{xy}$ is defined for the whole jet, based on the reconstructed SV of the charmed hadron, which is atypical of the currently used and planned summary variables. However, $d_0$ is defined for each individual track. Therefore, the mean of all the track $d_0$ values will be used as the distinguishing variable. This approach of calculating means will likely be the case for other distinguishing variables.

As can be seen in Figure \ref{fig:vertices}, $L_{xy}$ and $d_0$ are directly related to the displacement of the charmed hadron from the PV. As $L_{xy}$ is the transverse distance between the PV and SV, this is directly related to the hadron's lifetime. For $d_0$, as this is the distance of closest approach of the track to the PV, tracks from longer lived hadrons will on average have a larger $d_0$ due to them not originating from the PV.

As can be seen in Table \ref{tab:charm_props}, the lifetimes of the $D_s^+$ and the $D_0$ mesons are similar, within 100fs of each other. This means that purely relying on lifetime-related variables such as $L_{xy}$ and $d_0$ may not be sufficient to distinguish between jets containing those mesons. As a result, distinguishing variables related to the mass will also be investigated. 

\begin{figure}[ht]
    \centering
    \begin{subfigure}{0.48\textwidth}
        \centering
        \includegraphics[width=\linewidth]{images/lxy.png}
        \caption{Distribution of $L_{xy}$}
        \label{fig:lxy}
    \end{subfigure}\hfill
    \begin{subfigure}{0.48\textwidth}
        \centering
        \includegraphics[width=\linewidth]{images/d0_mean.png}
        \caption{Distribution of mean $d_0$}
        \label{fig:d0}
    \end{subfigure}
    \caption{Distributions of distinguishing variables for charmed hadrons. A sample of 1 million charm hadron events was generated with a minimum $p_T$ of 20 GeV.}
    \label{fig:charm_vars}
\end{figure}

\subsection{Impact Parameter Threshold for Mass-based Variables}

Mass-based variables require a more nuanced approach. The current approach for calculating lifetime-related summary variables is to take averages across all jet constituents, as the ATLAS data will not contain the truth information about which tracks originated from the charm hadron. The differing values of the lifetime-related variables will affect the averages across the whole jet, meaning this approach is valid. However, for mass-based variables, as the hadron mass is only a small fraction of the total jet mass, the effect on the average could be too small to detect. Hence, an alternative approach was taken, where tracks were filtered based on their $d_0$. Summary variables were calculated for tracks with a $d_0$ above a specific threshold, as those tracks are more likely to have originated from the charmed hadron. 

To find this threshold, a balance must be struck between signal efficiency, which is the fraction of the tracks correctly identified as coming from the charm hadron, and background rejection, which is the fraction of non-charm hadron tracks being identified as such. A larger signal efficiency leads to a lower background rejection rate, and vice versa, hence the need to strike a balance. To find this balance, signal efficiency and background rejection can be plotted on the same axes for different $d_0$ thresholds.

% ---------- FUTURE PLANS ---------- %

\section{Project Plan and Timeline}
The initial area of focus for the first two weeks after resuming the project will be implementing and testing the $d_0$ threshold approach for mass-based variables, and finetuning distinguishing variables. Then, two weeks will be dedicated to training an ML model, in the author's case a Feed Forward Neural Network (FFNN), and evaluating its performance. After this, two weeks will be spent simulating the ATLAS detector effects, retraining and reevaluating the model afterwards. The final two weeks will be spent applying the trained model to real ATLAS data and analysing the results. This gives a two week buffer period before the deadline to account for unforseen issues.

\subsection{Distinguishing Variable Development}
The approach of using $d_0$ thresholds to identify tracks originating from the charmed hadron will be further developed for mass-based variables, and will be applied to lifetime-based variables as well. The effectiveness of using this approach will be evaluated by comparing the performance of the machine learning model for the two different approaches. This will allow mass-based distinguishing variables to be incorporated. Variables that will be explored are the invariant mass of the tracks above the $d_0$ threshold, and the fractional $p_T$ of those tracks compared to the jet. 

\subsection{Machine Learning Model Development}
The development of the machine learning model will be split into two paths. The author will carry on with a simpler FFNN approach, continuing the current approach of using summary variables and further developing and fine-tuning those summary variables. In parallel, the project partner will develop a more advanced neural network architecture that can take in the four-momenta of the constituents directly, as opposed to using summary variables. These approaches will then be compared to determine their relative effectiveness.

\subsection{Detector Simulation}
In addition, the ATLAS detector will be simulated, possibly using DELPHES. This will allow detector effects, such as efficiency, resolution (via a Gaussian smear) and rejection of high pseudorapidity events, to be included in the simulation. This will improve the realism of the event generation, leading to more accurate training of the machine learning model.

\subsection{Application to ATLAS Data}
Once the MC simulated data is as realistic as possible, the trained ML model will be applied to real data from ATLAS. As this data contains experimental uncertainties, the transverse impact parameter significance, defined as $d_0$ divided by its uncertainty, will be evaluated and potentially incorporated. The ratio of $\Lambda_c^+$ baryons to D mesons in the ATLAS data will be then compared to that in the most realistic MC simulated data to investigate the universality of charm quark fragmentation.

% ---------- REFERENCES ---------- %

\bibliographystyle{plain}
\bibliography{references}

\end{document}